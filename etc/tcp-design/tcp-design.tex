\documentclass[11pt]{article}
\usepackage{listings}
\usepackage[english]{babel}
\usepackage{a4}
\usepackage{latexsym}
\usepackage[
    colorlinks,
    pdftitle={Design Document TCP Implementation},
    pdfsubject={TCP implementation},
    pdfauthor={Laurens Bronwasser, Martijn Vermaat}
]{hyperref}

\title{Design Document TCP Implementation}
\author{
    Laurens Bronwasser and Martijn Vermaat\\
    \{lmbronwa,mvermaat\}@cs.vu.nl
}

\begin{document}
\maketitle


\lstset{
  numbers=none,
  basicstyle=\small,
  frame=tb,
  language=C,
  captionpos=b
}


\section{High-level design}


The most important high-level design decision is that we choose to do a
three-tier implemenation. The three tiers in our TCP implementation are
the connection-ori\"ented tier, the state tier, and the connection-less
tier.

The connection-ori\"ented tier provides a TCP interface to the application
layer, whereas the connection-less tier deals only with sending and
recieving TCP packets and works directly on the IP layer interface.

The middle layer in our implementation will be the state tier. This tier
will have to bridge the gap between a connection-less environment and a
connection-ori\"ented environment, and therefore has to maintain a state.

\paragraph{}

Calling of procedures happens only within one tier, or from one tier to
the tier directly beneeth it. This idea can be better understood by
looking at the high-level design diagram attached.


\section{Low-level design}


\subsection{Connection-ori\"ented tier}


We will roughly sketch the structure of all procedures in the
connection-ori\"ented tier, mostly with pseudo-code, sometimes
augmented with comments.


\paragraph{}


\begin{lstlisting}[title=Procedure tcp\_read]
/*
  *buf      Buffer to store data
  maxlen    Maximum number of bytes to read

  Returns number of bytes read.

  This procedure is blocking.
*/

int tcp_read(char *buf, int maxlen) {

    while ( buffer_empty() ) {
        do_packet();
    }

    return copy_data(&buf, &maxlen);

}
\end{lstlisting}


\paragraph{}


\begin{lstlisting}[title=Procedure tcp\_write]
/*
  *buf      Buffer with data to write
  len       Number of bytes to write

  Returns 0 on success, number of bytes written on incomplete
  write, -1 otherwise.
*/

int tcp_write(char *buf, int len) {

    bytes_left = len;

    while (bytes_left) {
        /* send_data calls do_packet when needed */
        bytes_sent = send_data(*buf, bytes_left);
        if (bytes_sent == -1) break;
        bytes_left -= bytes_sent;
    }

    if (bytes_left == len) {
        /* will also happen if len=0 */
        return -1;
    } else if (bytes_left) {
        return len-bytes_left;
    } else {
        return 0;
    }

}
\end{lstlisting}


\paragraph{}


\begin{lstlisting}[title=Procedure tcp\_connect]
/*
  dst       Host address to connect to
  port      Port number on host to connect to

  Returns 0 on success, -1 on failure.
*/

int tcp_connect(ipaddr_t dst, int port) {

    /* send_syn calls do_packet when needed */
    return send_syn();

}
\end{lstlisting}


\paragraph{}


\begin{lstlisting}[title=Procedure tcp\_listen]
/*
  port      Local port to connect to
  *src      Address of client to listen to

  Returns 0.

  This procedure is blocking.
*/

int tcp_listen(int port, ipaddr_t *src) {

    set_state(LISTEN);
    do {
        do_packet();
        if (get_state() == SYN_RECEIVED) {
            send_syn();
        }
    while ( get_state() != ESTABLISHED );
    return 0;

}
\end{lstlisting}


\paragraph{}


\begin{lstlisting}[title=Procedure tcp\_close]
/*
  Returns 0, except when there is no active connection.
*/

int tcp_close() {

    /* check current state */

    /* send_fin calls do_packet when needed */
    return send_fin();

}
\end{lstlisting}


\paragraph{}


\begin{lstlisting}[title=Procedure tcp\_socket]
/*
  Returns 0 on success, -1 on failure.
*/

int tcp_socket() {

    return set_initial_state();

}
\end{lstlisting}



\subsection{State tier}


As we did in the previous section for the connection-ori\"ented tier, we
will describe most procedures of the state tier in this section, again by
using pseudo-code.
Additionally, we will describe the \verb|tcb| (TCP Control Block),
which contains the current state. Also, some methods to access information
on the current state from a higher tier will be described.


\subsubsection{Procedures}


\paragraph{}


\begin{lstlisting}[title=Procedure do\_packet]
/*
  Returns the number of bytes read, or -1 on failure.

  This procedure is blocking.
*/

int do_packet() {

    len = recv_tcp_packet(...., data);
    handle_ack(ack_nb);
    handle_fin(flags);
    handle_syn(flags);
    handle_data(data, data_sz);

    return len;

}
\end{lstlisting}


\paragraph{}


\begin{lstlisting}[title=Procedure handle\_ack]
/*
  ack_nb    Acknowledgement number

  Void procedure.
*/

void handle_ack(u32_t ack_nb) {

    /* Sequence numbers wrap around, solve this later */
    if (ack_nb > tcb.our_sequence_num
        && ack_nb <= tcb.expected_ack) {

        tcb.our_sequence_num = ack_nb;
        /* Adjust number of undelivered bytes */

        if (get_state() == SYN_RECEIVED) {

            set_state(ESTABLISHED);

        } else if (get_state() == SYN_SENT) {

            set_state(SYN_ACK_RECEIVED);

        } else if (get_state() == FIN_WAIT_1) {

            set_state(FIN_WAIT_2);

        } else if (get_state() == LAST_ACK) {

            set_state(CLOSED);

        } else if (get_state() == CLOSING) {

            set_state(TIME_WAIT);

        }

    }

}
\end{lstlisting}


\paragraph{}


\begin{lstlisting}[title=Procedure handle\_syn]
/*
  flags     TCP packet flags

  Void procedure.
*/

void handle_syn(u8_t flags) {

    if (contains_syn(flags)) {

        if (get_state() == LISTEN) {

            set_state(SYN_RECEIVED);
            /* send_syn will be called from tcp_listen */

            send_ack();

        } else if (get_state() == SYN_ACK_RECEIVED) {

            set_state(ESTABLISHED);

            send_ack();

        }

    }

}
\end{lstlisting}


\paragraph{}


\begin{lstlisting}[title=Procedure handle\_fin]
/*
  flags     TCP packet flags

  Void procedure.
*/

void handle_fin(u8_t flags) {

    if (contains_fin(flags)) {

        if (get_state() == ESTABLISHED) {

            set_state(CLOSE_WAIT);
            send_ack();

        } else if (get_state() == FIN_WAIT_1) {

            set_state(CLOSING);
            send_ack();

        } else if (get_state() == FIN_WAIT_2) {

            set_state(TIME_WAIT);
            send_ack();

        }

    }

}
\end{lstlisting}


\paragraph{}


\begin{lstlisting}[title=Procedure copy\_data]
/*
  buf       Char buffer to copy bytes to
  maxlen    Maximum number of bytes to copy

  Returns number of bytes copied.
*/

int copy_data(*char buf, int maxlen) {

    /* copy max <maxlen> bytes from buffer to <buf> */

}
\end{lstlisting}


\paragraph{}


\begin{lstlisting}[title=Procedure send\_data]
/*
  buf       Buffer with bytes to send
  len       Maximum number of bytes to send

  Returns number of bytes sent, or -1 on error.
*/

int send_data(char *buf, int len) {

    send = min(MAX_PACKET_SIZE, len);

    send_tcp_packet(..., &buf, send);
    tcb.our_sequence_number += send;
    tcb.expected_ack = tcb.our_sequence_number-1;

    do {
        do_packet();
    } while ( !<received ack> );

    return send;

}
\end{lstlisting}


\paragraph{}


\begin{lstlisting}[title=Procedure send\_syn]
/*
  Returns 0 on success, -1 otherwise.
*/

int send_syn() {

    send_tcp_packet(flags=syn,ack=current_ack);
    tcb.our_sequence_number++;
    tcb.expected_ack = tcb.our_sequence_number-1;
    /*set_state(X);*/

    do {
        do_packet();
    while (get_state() != ESTABLISHED);

}
\end{lstlisting}


\paragraph{}


\begin{lstlisting}[title=Procedure send\_fin]
/*
  Returns 0 on success, -1 otherwise.
*/

int send_fin() {

    send_tcp_packet(flags=fin,ack=current_ack);
    tcb.our_sequence_number++;
    tcb.expected_ack = tcb.our_sequence_number-1;
    set_state(FIN_WAIT_1);

    do {
        do_packet();
    while (get_state() == FIN_WAIT_1);

}
\end{lstlisting}


\paragraph{}


\begin{lstlisting}[title=Procedure send\_ack]
/*
  param     Description
*/

int send_ack() {

}
\end{lstlisting}


\subsubsection{TCP Control Block}


The \verb|tcb| (TCP Control Block) contains all information on the
current state. This includes client and host addresses and ports, buffer
data, packet sequence number, etcetera.


\paragraph{}


\begin{lstlisting}[title=The TCP Control Block structure]
typedef struct tcb_s {
    /* ... */
}
\end{lstlisting}


\paragraph{}


\begin{lstlisting}[title=Procedure buffer\_empty]
/*
  Returns whether or not the buffer is empty.
*/

int buffer_empty() {

    return tcb.bytes_in_buffer == 0;

}
\end{lstlisting}


\paragraph{}


\begin{lstlisting}[title=Procedure get\_state]
/*
  Returns current state.
*/

state_t get_state() {

    return tcb.state;

}
\end{lstlisting}


\paragraph{}


\begin{lstlisting}[title=Procedure set\_state]
/*
  state     State to set state to

  Returns 0 on success, -1 on failure.
*/

int set_state(state_t state) {

    tcb.state = state;

}
\end{lstlisting}


\subsection{Connection-less tier}


We will not write pseudo-code for the two procedures in this tier, because
we think their structure is not interresting enough. They simply send or
recieve a TCP packet.

The only two procedures in this tier are \verb|send_tcp_packet| and
\verb|recv_tcp_packet|.


\end{document}
