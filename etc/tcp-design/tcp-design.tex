\documentclass[11pt]{article}
\usepackage{listings}
\usepackage[english]{babel}
\usepackage{a4}
\usepackage{latexsym}
\usepackage[
    colorlinks,
    pdftitle={Design Documentation TCP Implementation},
    pdfsubject={TCP implementation},
    pdfauthor={Laurens Bronwasser, Martijn Vermaat}
]{hyperref}

\pagestyle{headings}

\title{Design Documentation TCP Implementation}
\author{
    Laurens Bronwasser and Martijn Vermaat\\
    \{lmbronwa,mvermaat\}@cs.vu.nl
}
\date{31 januari 2005}

\begin{document}
\maketitle


\lstset{
  numbers=none,
  basicstyle=\small,
  frame=tb,
  language=C,
  captionpos=b
}


\section{High-level design}


The most important high-level design decision is that we choose to do a
three-tier implementation. The three tiers in our TCP implementation are
the connection-ori\"ented tier, the state tier, and the connection-less
tier.

The connection-ori\"ented tier provides a TCP interface to the application
layer, whereas the connection-less tier deals only with sending and
recieving TCP packets and works directly on the IP layer interface.

The middle layer in our implementation will be the state tier. This tier
will have to bridge the gap between a connection-less environment and a
connection-ori\"ented environment, and therefore has to maintain a state.

\paragraph{}

Calling of procedures happens only within one tier, or from one tier to
the tier directly beneeth it. This programming paradigm is illustrated by the 
attached high-level design diagram.


\section{Low-level design}


\subsection{Connection-ori\"ented tier}


We will roughly sketch the structure of all procedures in the
connection-ori\"ented tier, mostly with pseudo-code, sometimes
augmented with comments.


\paragraph{}


\begin{lstlisting}[title=Procedure tcp\_read]
/*
  *buf      Buffer to store data
  maxlen    Maximum number of bytes to read

  Returns number of bytes read.

  This procedure is blocking.
*/

int tcp_read(char *buf, int maxlen) {

    while ( buffer_empty() ) {
        do_packet();
    }

    return copy_data(&buf, &maxlen);

}
\end{lstlisting}


\paragraph{}


\begin{lstlisting}[title=Procedure tcp\_write]
/*
  *buf      Buffer with data to write
  len       Number of bytes to write

  Returns 0 on success, number of bytes written on incomplete
  write, -1 otherwise.
*/

int tcp_write(char *buf, int len) {

    bytes_left = len;

    while (bytes_left) {
        /* send_data calls do_packet when needed */
        bytes_sent = send_data(*buf, bytes_left);
        if (bytes_sent == -1) break;
        bytes_left -= bytes_sent;
    }

    if (bytes_left == len) {
        /* will also happen if len=0 */
        return -1;
    } else if (bytes_left) {
        return len-bytes_left;
    } else {
        return 0;
    }

}
\end{lstlisting}


\paragraph{}


\begin{lstlisting}[title=Procedure tcp\_connect]
/*
  dst       Host address to connect to
  port      Port number on host to connect to

  Returns 0 on success, -1 on failure.
*/

int tcp_connect(ipaddr_t dst, int port) {

    /* send_syn calls do_packet when needed */
    return send_syn();

}
\end{lstlisting}


\paragraph{}


\begin{lstlisting}[title=Procedure tcp\_listen]
/*
  port      Local port to connect to
  *src      Address of client to listen to

  Returns 0.

  This procedure is blocking.
*/

int tcp_listen(int port, ipaddr_t *src) {

    set_state(LISTEN);
    do {
        do_packet();
        if (get_state() == SYN_RECEIVED) {
            send_syn();
        }
    while ( get_state() != ESTABLISHED );
    return 0;

}
\end{lstlisting}


\paragraph{}


\begin{lstlisting}[title=Procedure tcp\_close]
/*
  Returns 0, except when there is no active connection.
*/

int tcp_close() {

    /* check current state */

    /* send_fin calls do_packet when needed */
    return send_fin();

}
\end{lstlisting}


\paragraph{}


\begin{lstlisting}[title=Procedure tcp\_socket]
/*
  Returns 0 on success, -1 on failure.
*/

int tcp_socket() {

    return set_initial_state();

}
\end{lstlisting}



\subsection{State tier}

As we did in the previous section for the connection-ori\"ented tier, we
will describe most procedures of the state tier in this section, again by
using pseudo-code.
First, we will give an overview of the most important decisions that we made.


\subsubsection{Design decisions}


\paragraph{Buffer for incoming data}
    The buffer for incoming data is bigger than the maximum payload of a tcp
    packet. This is necessary because new data might come in while for example 
    \lstinline|tcp_write()| is polling for ack's. These data must be stored in the 
    buffer; 
    there is no way we can deliver it to the user.
    The buffer is circular because this prevents shoving of data to the begin of 
    the buffer to fill the gaps. 
    The state of the buffer is comprised in a variable that points to the 
    current start of the buffer, a variable that indicates the amount of data in 
    the buffer and of course the buffer itself.    

\paragraph{Pushing of incoming data}
    We don't assume that every incoming packet has the PSH flag set, eventhough 
    the assignment prescribes to set PSH on all outbound packets. That we do
    check the PHS flag means 
    that if it is set, \lstinline|tcp_read()| delivers is the data immediately.
    Only if there is no data to be pushed, \lstinline|tcp_read()| keeps on polling 
    ip for 
    more data until the amount of data specified by the user is received.
    
    \lstinline|tcp_read()| might return less (if the \lstinline|PSH| flag is set), but 
    the \lstinline|PSH| flag 
    will never cause \lstinline|tcp_read()| to return more 
    data than indicated by the users parameter.
    
    To keep track of the number of bytes that need to be pushed, a variable is 
    added to the tcb: \lstinline|rcvd_data_push|.

\paragraph{Acknowledgements}
    If data comes in tcp has never seen before (i.e. fresh data), an 
    acknowledgement is send
    immediately by \lstinline|handle_data()|. In real tcp implementations sending of
     ack's 
    should be postponed until the data is actually delivered to the user, 
    but since the assignment prescribes to use a window of size 1, we have to 
    acknowledge each packet instantly.

\paragraph{Connection establishment}
    After the first \lstinline|syn| is sent, the state transits to \lstinline|SYN_SENT| 
    and we wait for the \lstinline|syn+ack| response. In this state, only 
    \lstinline|syn+ack| packets will be accepted. If a packet arrives that carries an
    acknowledgement, but no \lstinline|syn|,...
    Als er een packet binnenkomt met een ack, maar zonder syn, dan
    negeren we dit packet. De SYN, waar we  op wachten, moet toch de
    verplichte ACK bevatten. Als het syn+ack packet niet binnen 1 seconde
    arriveert, dan gaat de state naar CLOSED, daarna wordt de syn opnieuw
    verstuurt. De state gaat dan weer terug naar \lstinline|SYN_SENT|.
    
\paragraph{State transitions}
    The \lstinline|declare_event()| procedure performs all necessary state 
    transitions, based on the event that is passed as a parameter to this 
    procedure. For example, 
    is the state is \lstinline|S_LISTEN| and a \lstinline|syn| is received, 
    \lstinline|E_SYN_RECEIVED| is passed to \lstinline|declare_event()|. The state will 
    then be altered to \lstinline|S_SYN_RECEIVED|. In this way, \lstinline|declare_event()| directly implements the state 
    transition diagrams shown in appendix A.
    
    There is a trade-off here between straightforward source code and source 
    code that has some build in bug-detection, which are both preferable. If we 
    had chosen for straightforward code, we had \lstinline|declare_event()| swallow
    every event that takes place. For example \lstinline|handle_ack()| would then 
    consist of only one line: 
    \lstinline|if (ack_flag) {declare_event(E_ACK_RECEIVED)}|. Then we leave it to 
    \lstinline|declare_event| to decide if a state transition is necessary.
    On the other hand, if
    some unexpected event happens, we wouldn't notice, because 
    \lstinline|declare_event()| gracefully discards the event, without any warning.
    
    To make it easier to detect unexpected events (caused by a bug), we decided 
    to make \lstinline|declare_event| accept only the events that are defined in the
    state transition diagram. On all other events it starts to yell (on the 
    screen). The consequence is that some code is a little bit more complex, 
    because we can't just say \lstinline|declare_event(E_ACK_RECEIVED)|
    on every incoming ack. We first have to check whether this ack event may 
    declared for a state transition.
    
    
\paragraph{Wrapping sequence numbers}
    The sequence number 
    
\subsubsection{State-tier Pseudocode}


\paragraph{}


\begin{lstlisting}[title=Procedure do\_packet]
/*
  Returns the number of bytes read, or -1 on failure.

  This procedure is blocking.
*/

int do_packet() {

    len = recv_tcp_packet(...., data);
    handle_ack(ack_nb);
    handle_fin(flags);
    handle_syn(flags);
    handle_data(data, data_sz);

    return len;

}
\end{lstlisting}


\paragraph{}


\begin{lstlisting}[title=Procedure handle\_ack]
/*
  ack_nb    Acknowledgement number

  Void procedure.
*/

void handle_ack(u32_t ack_nb) {

    /* Sequence numbers wrap around, solve this later */
    if (ack_nb > tcb.our_sequence_num
        && ack_nb <= tcb.expected_ack) {

        tcb.our_sequence_num = ack_nb;
        /* Adjust number of undelivered bytes */

        if (get_state() == SYN_RECEIVED) {

            set_state(ESTABLISHED);

        } else if (get_state() == SYN_SENT) {

            set_state(SYN_ACK_RECEIVED);

        } else if (get_state() == FIN_WAIT_1) {

            set_state(FIN_WAIT_2);

        } else if (get_state() == LAST_ACK) {

            set_state(CLOSED);

        } else if (get_state() == CLOSING) {

            set_state(TIME_WAIT);

        }

    }

}
\end{lstlisting}


\paragraph{}


\begin{lstlisting}[title=Procedure handle\_syn]
/*
  flags     TCP packet flags

  Void procedure.
*/

void handle_syn(u8_t flags) {

    if (contains_syn(flags)) {

        if (get_state() == LISTEN) {

            set_state(SYN_RECEIVED);
            /* send_syn will be called from tcp_listen */

            send_ack();

        } else if (get_state() == SYN_ACK_RECEIVED) {

            set_state(ESTABLISHED);

            send_ack();

        }

    }

}
\end{lstlisting}


\paragraph{}


\begin{lstlisting}[title=Procedure handle\_fin]
/*
  flags     TCP packet flags

  Void procedure.
*/

void handle_fin(u8_t flags) {

    if (contains_fin(flags)) {

        if (get_state() == ESTABLISHED) {

            set_state(CLOSE_WAIT);
            send_ack();

        } else if (get_state() == FIN_WAIT_1) {

            set_state(CLOSING);
            send_ack();

        } else if (get_state() == FIN_WAIT_2) {

            set_state(TIME_WAIT);
            send_ack();

        }

    }

}
\end{lstlisting}


\paragraph{}


\begin{lstlisting}[title=Procedure copy\_data]
/*
  buf       Char buffer to copy bytes to
  maxlen    Maximum number of bytes to copy

  Returns number of bytes copied.
*/

int copy_data(*char buf, int maxlen) {

    /* copy max <maxlen> bytes from buffer to <buf> */

}
\end{lstlisting}


\paragraph{}


\begin{lstlisting}[title=Procedure send\_data]
/*
  buf       Buffer with bytes to send
  len       Maximum number of bytes to send

  Returns number of bytes sent, or -1 on error.
*/

int send_data(char *buf, int len) {

    send = min(MAX_PACKET_SIZE, len);

    send_tcp_packet(..., &buf, send);
    tcb.our_sequence_number += send;
    tcb.expected_ack = tcb.our_sequence_number-1;

    do {
        do_packet();
    } while ( !<received ack> );

    return send;

}
\end{lstlisting}


\paragraph{}



\begin{lstlisting}[title=Procedure send\_syn]
/*
  Returns 0 on success, -1 otherwise.
*/

int send_syn() {

    send_tcp_packet(flags=syn,ack=current_ack);
    tcb.our_sequence_number++;
    tcb.expected_ack = tcb.our_sequence_number-1;
    /*set_state(X);*/

    do {
        do_packet();
    while (get_state() != ESTABLISHED);

}
\end{lstlisting}


\paragraph{}


\begin{lstlisting}[title=Procedure send\_fin]
/*
  Returns 0 on success, -1 otherwise.
*/

int send_fin() {

    send_tcp_packet(flags=fin,ack=current_ack);
    tcb.our_sequence_number++;
    tcb.expected_ack = tcb.our_sequence_number-1;
    set_state(FIN_WAIT_1);

    do {
        do_packet();
    while (get_state() == FIN_WAIT_1);

}
\end{lstlisting}


\paragraph{}


\begin{lstlisting}[title=Procedure send\_ack]
/*
  param     Description
*/

int send_ack() {

}
\end{lstlisting}


\subsubsection{TCP Control Block}


The \lstinline|tcb| (TCP Control Block) contains all information on the
current state. This includes client and host addresses and ports, buffer
data, packet sequence number, etcetera.


\paragraph{}


\begin{lstlisting}[title=The TCP Control Block structure]
/* TCP control block */
typedef struct tcb {
    ipaddr_t our_ipaddr;
    ipaddr_t their_ipaddr;
    tcp_u16t our_port;
    tcp_u16t their_port;
    tcp_u32t our_seq_nr;
    tcp_u32t their_seq_nr;  /* last byte we acked */
    tcp_u32t ack_nr;        /* the seq nr to ack in next packet */
    tcp_u32t expected_ack;  
    char rcv_data[BUFFER_SIZE];
    int rcvd_data_start;    /* pointer to start of circular buffer */
    int rcvd_data_size;     /* nr of bytes in buffer */
    int rcvd_data_psh;      /* number of bytes to push, (from start of buffer)*/
    char *unacked_data;     /* transmitted data yet to be acked (or resent) */
    int unacked_data_len;   /* length of transmitted data yet to be acked */
    state_t state;          /* stores the current state of the connection */
}

\end{lstlisting}


\paragraph{}


\subsection{Connection-less tier}

The only two procedures in this tier are \lstinline|send_tcp_packet()| and
\lstinline|recv_tcp_packet()|.

\paragraph{Procedure recv\_tcp\_packet()}
This procedure calls \lstinline|ip\_receive()| only once. If a packet is corrupted,
or if it is no tcp packet at all, \lstinline|recv\_tcp\_packet| returns -1.
That it does not loop to wait for a valid packet, is because this might take 
forever. The looping is taken care of by \lstinline|tcp\_read()|, which checks if the alarm 
went of on every cycle.



\end{document}
