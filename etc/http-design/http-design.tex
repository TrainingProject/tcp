\documentclass[11pt]{article}
\usepackage{listings}
\usepackage[english]{babel}
\usepackage{a4}
\usepackage{latexsym}
\usepackage[
    colorlinks,
    pdftitle={Design Documentation HTTP client and server},
    pdfsubject={HTTP client and server},
    pdfauthor={Laurens Bronwasser, Martijn Vermaat}
]{hyperref}

\pagestyle{headings}

\title{Design Documentation HTTP client and server}
\author{
    Laurens Bronwasser and Martijn Vermaat\\
    \{lmbronwa,mvermaat\}@cs.vu.nl
}
\date{31 januari 2005}

\begin{document}
\maketitle


\lstset{
  numbers=none,
  basicstyle=\small,
  frame=tb,
  language=Pascal,
  captionpos=b
}


\section{General remarks}


\subsection{HTTP messages}\label{sec:httpmessages}

The HTTP/1.0 specification states that a full HTTP message consists of a
message header, followed by a header/body separator, followed by a message
body. The message body is ended by closing the connection. This has some
implications for both HTTP clients and servers we will discuss below.


\subsection{Be liberal in what you expect \ldots}

On several occasions, the HTTP specification notes that ideal implementation
should try to accept messages that are only invalid because of trivial
errors (if the results is unambiguous). For instance, each header line should
be terminated by a carriage return and a newline. Implementations should be
liberal in what the expect, in the sense that they should try to accept
messages where header lines are only terminated by a newline.

Although not everyone agrees with this approach, we do think it is a good way
to build communicating programs. However, to keep our implementation as simple
as possible, we did not go through extra trouble especially to be liberal in
what we expect. So in a lot of cases, the HTTP server and clients are strict
conforming to the specification, but in some cases they try to be nice to the
other application.


\subsection{\ldots and conservative in what you send}

In contrast with the liberal attitude one should have on accepting incomming
data, it is never a good thing to have the same attitude on the assembling of
data you send, as this might break the other application. In both the HTTP
client and server, we tried to conform with the specification as strict as
possible. When in doubt, we are being conservative.


\subsection{Shared methods and header file}

For the HTTP client and server, we were allowed to use shared files
\lstinline|http.c| and \lstinline|http.h|. In the end, we decided not to use
them. Although there are a small number of methods and some type declaration
we could share between the client and server, the amount is not significant
enough in our opinion to be put in a separate file. Furthermore, some of this
code hasn't got anything to do with HTTP in specific, so putting it in
\lstinline|http.x| would be entirely good practice.


\section{HTTP client}

In the HTTP client, the central methods are \lstinline|do_request| and
\lstinline|handle_response|. Both are called in the \lstinline|main|
method. The HTTP \lstinline|GET| request is assembled and sent in the
\lstinline|do_request| method, while the the HTTP response message is read and
processed in the \lstinline|handle_response| method. The latter checks the
HTTP response status code and writes the message body to a file if it was a
status of ``200 OK''.


\subsection{General control flow}

The general control flow in the HTTP client is described in psuedo-code in the
following listing:

\paragraph{}

\begin{lstlisting}[title=HTTP client control flow]
main(url):
    ip, filename := parse_url(url)
    connect(ip)
    do_request(filename)
    handle_response(filename)
    close()

handle_response(filename):
    header = get_response_header()
    if (status_ok(header)):
        body = tcp_read()
        write_to_file(filename, body)
    else:
        print_error()
\end{lstlisting}


\subsection{Retrieving the response message}

\paragraph{Message header}

Since the message header is always followed by a blank line separator, the
HTTP client reads incomming data until it finds this separator. If it isn't
find, we can take this as an error in the response message.

\paragraph{Message body}

The message body, however, is not followed by any kind of marker. When reading
the message body, we know we have it complete if we reach the end of the
incomming data stream. Therefore, we read the entire stream and have a
complete message body if the end of the stream is reached. We also don't
depend on the value of the \lstinline|Content-Length| HTTP header, because it
might not be correct, or be not even present at all.

If we can't reach the end of the incomming data stream, this is an error. The
partial data will still be written to a file, but the user will be notified of
this.


\subsection{Data buffers}\label{sec:clientbuffers}

In the client, we have one central buffer, the response buffer. Assembling the
client HTTP request is not much work, the buffer used for it is local in the
\lstinline|do_request| method.

The response buffer is used by much more methods, but we choose not to
implement it as a global buffer. Instead, it is initialized in the
\lstinline|handle_response| method which passes a pointer to our current
location in the buffer and the remaining size of the buffer. During processing
of the response message body, new data is read from the server and put in the
buffer whenever the entire buffer is processed. This is done until there is no
more data to read; the end of the stream is reached.

During processing of the response message header, no further copying of data
from the network to the buffer is done. This implies that the response message
header should always fit in the response data buffer. Although this is
certainly a shortcomming of the implementation, we think it is safe to assume
the response message header will never be more than, say, two kilobytes.


\section{HTTP server}

The HTTP server is built around a single \lstinline|serve| method that can either
be called once, or an infinite number of times in a loop. The \lstinline|serve|
methods listens on a defined port for incomming connections and reads an HTTP
request message when a connection is established. If the request message can
be parsed correctly, the relevant response message will be constructed and
sent to the client. After that, the connection will be closed and \lstinline|serve|
will be called again, or the HTTP server will exit.


\subsection{General control flow}

In the listing below we describe the general control flow in the HTTP server:

\paragraph{}

\begin{lstlisting}[title=HTTP client control flow]
main():
    while true: serve()

serve():
    listen()
    request := get_request()
    filename := parse_request(request)
    write_response(filename)
    close()

write_response(filename):
    write_headers()
    write_body(filename)
\end{lstlisting}

The \lstinline|while| loop in the \lstinline|main| method of the server can
also be one call to the \lstinline|serve| method, depending on a preprocessor
instruction. Default is not to loop.


\subsection{Sending the response message}

As noted in subsection \ref{sec:httpmessages}, the message body will be
considered completely sent if the connection is closed. Therefore, we should
not explicitely close the connection if sending of the message body
failed (a client might think everything went well, while the body was only
partially transfered). Because we don't send the message header and body with
separate write actions, we never close the connection if sending of data
failed (although following the spec, it may be done if the message header was
not sent).


\subsection{Data buffers}

\paragraph{Request message data}

In our HTTP server, the buffer for the request message data is a local buffer
in the \lstinline|server| method. It is filled once with data from the client,
and then processed. Just like we noted about the response buffer in the HTTP
client (subsection \ref{sec:clientbuffers}), this means the full HTTP request
message header should always fit in the response buffer.

\paragraph{Response message data}

The response message data is stored in a global buffer. We decided to use a
global buffer because a lot of methods make references to it and it is needed
during almost the entire lifetime of the server execution. Data to be sent to
the client is written to the buffer with the \lstinline|write_data|
method. This method checks if there is enough free space left in the buffer,
and if not, it first sends the data in the buffer to the client and tries
again. This way, there is no limit on the amount of data we can send, because
it will not have to be stored in a buffer all at the same time.


\subsection{Running as superuser}

\paragraph{\lstinline|chroot|-ed environment}

Running a process like a HTTP server with superuser priviliges is not
recommended. If our server is executed with these priviliges, it will try to
add extra safety by creating a \lstinline|chroot|-ed environment of the
directory containing the documents to be served. Bugs in the filehandling code
can therefore never lead to unforeseen exposure of private data outside this
directory.

\paragraph{The \lstinline|other| read bit}

In addition, before opening a file requested for by a client, the server will
check the read permissions of the file. If the \lstinline|other| read bit is
not set, the server will not open the file and return a ``404 Not Found''
response message.


\subsection{Other HTTP methods}

Besides the \lstinline|GET| method, HTTP/1.0 defines several other
methods. Our server responds to these requests with a ``501 Not Implemented''
response message, because indeed, the only method implemented is the
\lstinline|GET| method. Implementing some of the other methods, however, is
trivial. For example, just add a \lstinline|add_put| method to handle
\lstinline|PUT| requests on which the \lstinline|write_response| method can
dispatch.


\section{Testing the implementations}

\paragraph{Automated testing}

For the HTTP client and server, we have not automated test suite. This was not
really a deliberate decission of ours, but rather the result of too little
time. However, since the functionality of the server and especially the client
is pretty limited, in the end we think manual testing was good enough for us.

\paragraph{Manual testing}

To test if files were transfered correctly, we calculated MD5 checksums of the
files before and after sending and compared the results. To simulate an
unreliable network, we used libip\_udp by Marten Klencke and Erik
Bosman. libip\_udp is a simulation of the IP implementation provided on Minix
for the Linux environment. It supports easy testing with different amounts of
packetloss.


\end{document}
